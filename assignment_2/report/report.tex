%\documentclass[english, a4paper]{article}
\documentclass{llncs}
\usepackage{llncsdoc}

\usepackage{float}
\usepackage[pdftex]{graphicx}
\usepackage[font={small,it}]{caption}
\usepackage[caption=false]{subfig}
\usepackage{url}
\usepackage{siunitx}
\usepackage{graphicx}
\usepackage{pbox}
\usepackage{placeins}

\newcommand{\squeezeup}{\vspace{-8.9mm}}
\setcounter{secnumdepth}{3}
\addtolength{\textfloatsep}{-3mm}
\addtolength{\belowcaptionskip}{-15pt}



\title{Data Mining -- Assignment 2}

\author{Andrew Bedard (2566978) -- Artagan Malsagov (2562231)  -- Shabaz Sultan(2566703)}

\institute{}
\begin{document}
\maketitle
\section{Introduction}
The online travel agency (OTA) Expedia posed to researchers the challenge to rank hotels by their likelihood of being booked. In such a competitive market as the one of click-through purchases, properly ranking offered hotels according to user's preferences becomes indispensable if the OTA wishes to win the sale. See the Kaggle platform \cite{WinNT} for further information.

This report explains our approach to tackling this problem based on the search and click-through data provided by Expedia. First, a description of the data is provided and a more rigorous definition of the problem is given. Then follows a review of the methods and models used and how they held up to the challenge. Finally, we finish off with a summary and some concluding remarks.  

  
\section{Background}


          
\bibliographystyle{plain}
\bibliography{report}
\end{document}
