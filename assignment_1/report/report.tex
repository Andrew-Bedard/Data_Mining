\documentclass[english, a4paper]{article}

\title{Data Mining -- Assignment 1}
\author{Andrew Bedard -- Artagan Malsagov -- Shabaz Sultan}
\begin{document}
\maketitle
\section{Titanic Survivors\\ \large or: How I learned to Stop Worrying and Love the Data}
\subsection{Introduction}
On the 14th of April, 1912 the RMS Titanic hit an iceberg and sank a few hours later. Of the 2207 people on that ship 1501 lost their lives on that night. These tragic events offer the opportunity to study exactly how people behave during such a life-and-death event.\\
From an economic perspective one can wonder if the model of humans as `homo economicus', that of humans as rational, self-interested actors is the best way to model behaviour of people on that night. An econometric analysis shows that such a `homo economicus' model is overly simplistic because females and children were more likely to survive than physically stronger males. An analysis using said model is still valuable however because people who were closer to their prime age, were of higher social class or had access to more information were more likely to survive\cite{Frey2010}\cite{Frey2011}. \\
The reason why the deathrate was so high starts with the fact that there were not enough lifeboats. On top of that fewer passengers survived because they of supposed reluctance to leave the ship (e.g. because of disbelief that the ship would actually sink or wives that did not want to be separated from their husbands). The difference in survival between males and females can be explained as a result of policy. The official explanation from the Mersey inquiry to explain the difference in survival rates between classes was that lower-class people were less willing to be parted from their belongings and that their English was poorer making them less able to follow orders from the crew. Statistical analysis based on nationality as a proxy for language ability refutes this second claim and suggest that explanations rejected by the inquiry (layout of the ship disadvantaging lower class people and outright discrimination when letting people on lifeboards) are more likely \cite{Hall1986}.

\bibliographystyle{plain}
\bibliography{report}
\end{document}
